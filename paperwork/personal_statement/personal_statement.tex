%! Author = breandan
%! Date = 6/7/21

% Preamble
\documentclass[11pt]{article}

% Packages
\usepackage{amsmath}
\usepackage{unicode-math}
\usepackage{amssymb}

\usepackage{float}
\usepackage{tikz}

% Document
\title{Autobiographical Statement}
\author{Breandan Considine}
\begin{document}
\maketitle
I was born July 30th, 1990 in Hanover, New Hampshire. As an only child, my parents divorced at an early age, and I was raised primarily by my maternal grandparents in Bethlehem, New Hampshire. My mother worked as a bank examiner and traveled often for work. My grandmother, raised me while my mother was away, and taught me to love Jesus. Some of my earliest memories were praying the rosary with her before bed. She was a catechism teacher for the local parish and made a home for us, together with my grandfather, my Uncle Mark and I.

At the age of eight, I was transferred from the public school in Bethlehem to a Catholic School in St. Johnsbury, Vermont. The school was a half an hour from home, and my grandmother would drive an hour twice a day to drop me off and pick me up after school. Later, a commuter bus started between St. Johnsbury and Littleton, and I would ride the bus with the workers to school. I was an avid reader, and loved to draw.

After graduating eighth grade, I was enrolled in a private school, St. Johnsbury Academy, where I excelled in languages. I studied Latin and Chinese and took an interest in Java programming and robotics. One of my earliest programming memories was using a TI-83 calculator to generate prime numbers using the Sieve of Eratosthenses. I fell in love with programming, which became a lifelong passion.

After high school, I went on to the Rochester Institute of Technology, where I studied computer science and participated in the sailing club. During the summers, I worked at a small company in New Hampshire doing data analysis. I also spent year studying abroad in Shanghai, China. One day while attending Mass there, I ran into an old classmate who, unbeknownst to me, was studying there at the same time.

This brief experience has reverberated in different times and similar ways over the years, but this was my first memorable encounter with the Holy Spirit. That -- and similar experiences throughout my life have reminded me that He is alive and at work in the world.

After graduating from college, I worked as an engineer developing automated bidding software at an ad tech startup in Austin, Texas. Following that, I worked for two years as a technical evangelist for a software engineering company, where I had the opportunity to travel to conferences around the world teaching programming and software engineering. I gave talks about speech recognition, developer tools and machine learning.

I returned to graduate school in 2017 to pursue machine learning, which was at the time making significant strides in research and industry. After being admitted to the Masters degree program at University of Montreal, I helped automate the software infrastructure for the AI driving competition. Afterwards, I continued as a Ph.D. student at McGill University to work on machine learning for software engineering.

During my graduate studies, I had the opportunity to teach and loved working with students on research problems. I enjoyed the social aspect of organizing events together with my colleagues, through academic seminars, workshops and conferences. As a teaching assistant for several graduate courses in machine learning, I enjoyed working in fast-paced learning environments with motivated people on challenging problems.

Another one of the joys I experienced during graduate school was the joy of understanding. As a researcher, you spend years stumbling around in the dark, cursing and toiling over some wretched problem, chasing some elusive idea through dark and dingy alleys. Then, when you least expect: a flash of inspiration - lightning strikes! Research, for me, is long periods of uncertainty punctuated by fleeting moments of clarity. Like meeting a long lost friend in a strange land, or unraveling a clue that has long haunted you but failed to reveal its meaning. These moments make all the confusion, doubt and toil worthwhile and worth a hundredfold more.

One of the difficulties I have faced as a researcher is learning to keep the ego at bay. Academia celerbrates intellectual achievement -- worships it even. This encourages healthy competition and keeps us honest to a degree, but -- I have come to realize -- can feed the ego and lead to false idols.

During the course of pursuing higher education, I lost my connection to the faith. I stopped attending Mass, and began to trust in machines. With the advent of intelligent machines and the apparent triumph of computationalism, I began to question the meaning of life and what humans are meant to do here on earth. After this dark period, I have come to realize the truth of what I was taught long ago: we are born sinners, and redeemed by Christ's sacrifice. We are called to follow His example, by giving of ourselves to others. Neither artificial intelligence, nor sin, nor forbidden fruit of the tree of knowledge changes this.

I first heard the call to religious vocation, when I began to realize all these experiences are meaningful: what we are seeking is not intelligence or epiphany or serendipity or insight, but the kingdom of heaven itself. Glimpses from across the shroud, of another world. Like Matthew the tax collector, who meets Jesus of Nazereth, leaves his posessions behind and follows him. Or the merchant (Matthew 13:45-46), who upon finding the pearl of great price, sells all his belongings to purchase it. Like Ambraham, who hears God's voice and offers up his only son in sacrifice. What is ours is not ours, but given by God, and to Him we must freely give in return.

Recently, as I reflect on the capriciousness of my own life, I am growing conscious of the tremendously fortunate circumstances I have enjoyed, made possible by the sacrifices of others. I am filled with gratitude for the opportunities I was given, and in many ways troubled by the enormous debt of goodwill I have accumulated, to which qualification I cannot attribute. I hope to spend the remainder of my life repaying this debt, using what little knowledge and wisdom I have earned to help others.

Each chapter of my life has been uniquely privileged. I am tremendously fortunate to have received many undeserved chances to pusue my interests. I feel an enormous responsibility to repay God for the many gifts I have received and in some ways squandered on frivolous pursuits. For this reason, I am applying to the seminary to follow the call and become a tool in His service. I am moved to praise the Lord God Almighty, to teach, and to give back to those from whom I have received so much.

During a recent visit to Moreau Seminary, I had the chance to meet with Fr. DeRiso and Fr. Gallagher and expressed to them that I had received God's call. Frankly, I am still somewhat mystified that He would call me, a computer science student. But if God wills it, I will go. Thank you for considering my application.
\end{document}