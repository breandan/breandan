%! Author = breandan
%! Date = 6/7/21

% Preamble
\documentclass[11pt]{article}

% Packages
\usepackage{amsmath}
\usepackage{unicode-math}
\usepackage{amssymb}

\usepackage{float}
\usepackage{tikz}

% Document
\title{Statement of Intent}
\author{Breandan Considine}
\begin{document}
\maketitle
Broadly, I am interested in computer science as it informs the human condition. During my graduate studies, I explored the connection between pragmatic language games and their relation to human-computer interaction in software engineering. At a high level, my doctoral research has two main components: cooperative learning in integrated developer environments, and the relation between syntax and structure in automated program repair.

My research over the last four years has primarily focused on program repair, using the problem of syntax error correction as a vehicle to study the interplay between learnability, composability and expressive power in formal languages. In this work, we pose repair as a two-player game between a programmer and an editor, reframing admissibility in terms of language intersection, then benchmark the efficiency of pretrained large language models at subsampling the admissible set. This can also be seen as a form of constrained decoding or weighted model counting over formal languages.

Not only do we use the problem of syntax repair to investigate the limits and sample efficiency of language models, our research also seeks to build practical developer tools for software engineers in popular programming languages, such as Python and Java. As a prototype, we have implemented a realtime automated syntax repair engine that leverages incrementality and parallelism to continuously monitor and suggest plausible syntax repairs.

In a second line of work, I am interested in certain design and implementation questions that arise in the development of differentiable and probabilistic programming languages. During my Master's thesis, I worked on a type-safe domain specific language for automatic differentiation and typelevel programming methods for checking tensor arithmetic using parametric polymorphism with mixed site variance, as found in the Kotlin type system.

Towards the end of my Ph.D., I grew interested in the rich tapestry weaving together philosophy, theology and computer science. In recent work, I am surveying some of the historical connections between the field of computer science and the Catholic intellectual tradition. More broadly, I am interested in Christian anthropology as it relates to artificial general intelligence, the virtue ethics of autonomous systems, and the escatological significance of technological singulitarianism. Enclosed in my application to the M.Div. program is a preprint of this survey that retraces how Catholic scholars and theologians have influencened the develoment of computer science.

In the future, I hope to lay the foundation for a more rigorous approach to human-centered software engineering that unifies formal languages with human factors and qualitative methods, such as grounded theory. I am also devoted to expanding and sharpening our understanding of thinking machinery through the lens of Catholic social teaching. Through my Ph.D. studies, I have come to believe that a more holisitic liberal arts education in computer science and an enhanced technical focus on verifiability and mathematical rigor in the social sciences would be enriching for both disciplines.
\end{document}