%! Author = breandan
%! Date = 6/7/21

% Preamble
\documentclass[11pt]{article}

% Packages
\usepackage{amsmath}
\usepackage{unicode-math}
\usepackage{amssymb}

\usepackage{float}
\usepackage{tikz}

% Document
\title{Statement of Intent}
\author{Breandan Considine}
\begin{document}
\maketitle
%Broadly, I am interested in computer science as it relates to the humanities.
During my Ph.D. studies, I explored the connection between pragmatism in the 21st century and its connections to modal logic and human-computer interaction in software engineering. At a high level, I am interested in two research tracks: cooperative learning in stochastic langauge games, and the relation between syntax and structure in automated program repair.

My research over the last four years has primarily focused on syntax error correction, which uses the problem of program repair as a vehicle to study the interplay between learnability and expressive power in formal languages. Our works poses syntax repair as a two-player game between a programmer and an editor, to predict authorial intent by reframing admissibility in terms of language intersection, then benchmarks the efficiency of stochastic language models to subsample the admissible set. This can be seen as a form of constrained decoding or weighted model counting over formal languages.

%In addition to studying the limitations and sample efficiency of neural language models on repeated language games, our research on syntax repair is being used to build practical tools for software engineers in downstream programming languages, such as Python and Java.

In addition to our work on automated reasoning for software development, I am also interested in the design and implementation of differentiable and probabilistic programming languages. During my Master's thesis, I worked on a type-safe domain specific language for automatic differentiation, and type-level programming methods for embedding shape-constraints using parametric polymorphism, as found in the Kotlin type system.

Towards the end of my Ph.D., I began to grow aware of the rich and varied connections between philosophy, theology and computer science. In more recent work, I am surveing the historical origins of machine learning in the Catholic intellectual tradition. I am interested in retracing the semiotic and heremenutic origins of artificial intelligence and how Catholic theologians influencened the development of computer science. Enclosed in my application to the M.Div. program is a preprint of this survey.

In the future, I hope to lay the foundation for a more rigorous study of human-centered software engineering, by unifying algorithmic information theory and formal methods with human factors and qualitative methods, such as grounded theory in the social sciences.
\end{document}