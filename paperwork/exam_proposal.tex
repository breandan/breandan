%! Author = breandan
%! Date = 11/16/20

% Preamble
\documentclass[11pt]{article}

% Packages
\usepackage{amsmath}
\usepackage{amsfonts}
%\usepackage{prooftrees}
\usepackage{bussproofs}
\usepackage{natbib}
\usepackage{float}
\usepackage{xcolor}
\pagecolor[rgb]{0.1,0.1,0.1} %black
\color[rgb]{0.7,0.7,0.7} %grey



\title{Comprehensive Exam Syllabus}
\author{Breandan Considine}
\date{\today}

% Document
\begin{document}
    \maketitle

    \tableofcontents
    \pagebreak

%    TODO: think of a good title
%
%    \begin{enumerate}
%      \item Pattern recognition in structured documents?
%      \item Towards a notion of sameness in software artifacts
%      \item Computer-Aided Programming...
%      \item Knowledge-driven development
%      \item Content recommendation for programming tools
%      \item Reasoning and relating software artifacts
%      \item Traceable documentation retrieval
%      \item Entity matching for programming recommendations
%    \end{enumerate}

    \section{Introduction}

    Historically, most knowledge was encoded as natural language. A growing fraction is now software~\citep{allamanis2018survey}. Software shares many statistical properties in common with natural language~\citep{hindle2012naturalness}. But software is also a regular language which can be described analytically. Unlike natural language, software has a deterministic grammar with explicit natural and denotational semantics~\citep{pierce2010software}.

    Prior work~\citep{considine2019kotlingrad,considine2019programming} explored differentiable software. Differentiability was a key insight for the development of learning systems. But it does not provide the vocabulary to describe uncertainty, a prerequisite for expressing human knowledge. In order to reason as humans do, software must be able to integrate human knowledge in the form of software. In this work, we propose a set of tools to help humans reason about software.

    In order to reason about software, we need to define what it means for two software artifacts to be equal. The notion of equivalence has been studied in many logical disciplines. Various algorithms have been proposed to determine whether two objects are ``equal'', for some definition of equality and with respect to some set of axioms. We can poke and prod them, rewrite them, and compare them in various contexts to ascertain their equivalence.

    Ultimately, strict equivalence is too rigid to work in practice. A more robust definition would allow us to compare two objects in the presence of naturally-occurring stochasticity. What is the likelihood they came from the same distribution? How important are their differences? To answer these questions, we need a language capable of describing uncertainty. This work falls under the umbrella of probabilistic programming and reasoning.

    Identifying sameness is an important problem in knowledge systems known as entity alignment. For example, we can recycle software in a large codebase to improve code sharing. We can retrieve code samples and documentation to assist developers writing and debugging code. We can compose functions over data to synthesize programs with some desired properties. As we will see, all these require a foundation for probabilistic reasoning.

    In the following document, we will define exact and approximate equality in the deterministic and probabilistic setting, how to decide it exactly and approximately using rewriting and equivalence testing, then describe some applications for software engineering, code search and program repair.

    \pagebreak

    \section{Defining Equality}

    The $=$ notation is highly overloaded in mathematics and computer science. Roughly, it tells us when we can replace, or are replacing one thing with another. Two expressions are considered equal if, loosely speaking, they represent the same value. It has the following logical properties:

    \begin{prooftree}
        \AxiomC{}
        \UnaryInfC{$a = a$}
        \DisplayProof
        \hskip 1.5em
        \AxiomC{$a = b$}
        \UnaryInfC{$b = a$}
        \DisplayProof
        \hskip 1.5em
        \AxiomC{$a = b$}
        \AxiomC{$b = c$}
        \BinaryInfC{$a = c$}
        \DisplayProof
        \hskip 1.5em
        \AxiomC{$a = b$}
        \AxiomC{$F: A \rightarrow *$}
        \BinaryInfC{$F(a) = F(b)$}
    \end{prooftree}

    The problem is, mathematical equality is not a differentiable operator.
    If we wanted to write a differentiable program using the expression $\texttt{if(i = j)}$, we are forced to use a definition which is not entirely sound. Consider the Kronecker delta function $\delta_k: \mathbb{T}$, where $\mathbb{T} \in \{\mathbb{Z, Q, B}\}$:

    $$
    \delta_k(i, j) :=
    \begin{cases}
        1 \text{ if } i = j, \\
        0 \text{ otherwise }\\
    \end{cases}
    $$

    This is not a continuous function: the preimage of an open set is not open. Thus, it is not differentiable. Maybe a way to fix it, e.g. CADLAG. Now consider the Dirac delta function, $\delta_d: \mathbb{R} \rightarrow \mathbb{R}$. Although it cannot be written down directly, it can be described indirectly:

    $$
    \int_{-\infty}^{\infty} \delta_d (x)dx = 1 \text{ and } \delta_d(i, j) :=
    \begin{cases}
        \infty \text{ if } i = j, \\
        0 \text{ otherwise }\\
    \end{cases}
    $$

    Do we really need to use $\infty$ just to describe equality? This is a hack.

    Both of these are idealized concepts which do not exist in the real world. Unknowable whether the universe is discrete or continuous. Most software is discrete. All we can do is approximate and compare empirically.

    The fact that many humans have converged to these definitions captures an aspect of the human brain that we would expect a synthetic reasoner to also possess. To allow a more flexible version of the $=$ operator, we require a notion of equality which approximates the logical properties of $\delta_k$, but which can be made differentiable like $\delta_d$. (Basically, we need a metric, e.g. earthmover, Kantorovich-Rubinstein, Prokhorov et al.)

    \subsection{Intensional equivalence}\label{subsec:intensional-equivalence}

    Let $R: (\mathcal{I} \rightarrow \mathcal{O}) \times (\mathcal{I}\rightarrow \mathcal{O})$ be a relation between functions which is closed under composition. We say that $f_1$ and $f_2$ are intensionally equal if $R_n(f_1)=R_m(f_2)$ for some $m,n \in \mathbb{Z}$. For example, suppose we have two functions $f_1, f_2: \mathbb{Z}^3 \rightarrow \mathbb{Z}$ where $f_1(x, y, z)=xz + xy$ and $f_2(x, y, z)=x(y + z)$, and $\mathcal{R}={a + b := b + a, a(b + c) := ab + ac}$ is our rewrite system. If we apply $R$ twice to $f_2$, we obtain $R_2(f_2)=x(y + z):=xy + xz:=xz + xy=R_0(f_1)$ and thus $f_1$ and $f_2$ are both intensionally equal.

    \subsection{Extensional equivalence}\label{subsec:extensional-equivalence}

    We say $f_1$ and $f_2$ are extensionally equivalent if $\forall i \in \mathcal{I}, f_1(i)=f_2(i)$, or in other words, ``Do $f_1$ and $f_2$ behave in the same way over all inputs?" While extensional equality is difficult to show if $|\mathcal{I}|$ is large, detecting inequality for pairs of random functions is relatively easy: we can simply search for $\hat i \in \mathcal{I}$ such that $f_1(\hat i) \neq f_2(\hat i)$. For example, suppose we have two functions $f_1, f_2: \mathbb{Z}^3 \rightarrow \mathbb{Z}$ where $f_1(x)=-x^3$ and $f_2(x)=|x|^3$. If we test $\hat i \in {-2, -1, 0, 1, \ldots}$, we have $f_1(-2)=f_2(-2)$, $f_1(-1)=f_2(-1)$, $f_1(0)=f_2(0)$, $f_1(1) \neq f_2(1)$. Once we detect an $f_1(\hat i) \neq f_2(\hat i)$, we can halt immediately.

    \subsection{Observational equivalence}

    Two terms $\textt{M}$ and $\textt{N}$ are observationally equivalent iff $\forall \textt{C[...]}$ where $\textt{C[M]}$ is valid in our language and halts, $\textt{C[N]}$ is also valid and halts. This is kind of like word embeddings.

    $P(w_t = a | w_{t-1}, w_{t-2}\ldots, w_{t+1}, w_{t+2}, \ldots)\overset{?}{\approx} P(w_t = b | w_{t-1}, w_{t-2}\ldots, w_{t+1}, w_{t+2}, \ldots)$

    We can define semantic equality in our setting using word embeddings.

    \section{Deciding Equality}\label{sec:structural}

    Various methods have been proposed to decide exact and approximate equality in the deterministic and probabilistic setting.

    {\centering
    \begin{table}[H]
        \begin{tabular}{c|c|c|}
            \cline{2-3} & Deterministic & Probabilistic \\ \hline
            \multicolumn{1}{|l|}{Exact}       & \begin{tabular}[c]{@{}l@{}}Types\\ Rewriting\end{tabular}   & \begin{tabular}[c]{@{}l@{}}Probabilistic circuits\\ Variable elimination\end{tabular} \\ \hline
            \multicolumn{1}{|l|}{Approximate} & \begin{tabular}[c]{@{}l@{}}Testing\\ Debugging\end{tabular} & \begin{tabular}[c]{@{}l@{}}Sampling\\ Simulation\end{tabular}                         \\ \hline
        \end{tabular}
    \end{table}
    }

    So far, we have described the properties of equality, but we still need an algorithm for how to establish it. We need a decision algorithm.

    It is seldom the case that two equal things are trivially equal. We need to do some computation to put them into a form that makes their equality plain to see.

    For exact equality, we need some notion of a transformation or rewrite. Equivalence requires:

    1. reducing an object into a normal form (canonicalization), then

    2. doing some kind of trivial comparison (e.g. string comparison)

    For approximate equality, we can just generate some representative data and test away.

    \subsection{Equality is undecidable}

    Mathematics is often a useful approximation of reality, but many mathematical concepts are impossible to realize. This is due to two problems:

    \begin{enumerate}
        \item intrinsic: the mechanics of the system are intrinsically unsound or impossible to mechanize. How do we encode $\lim_{x \rightarrow \infty}$?
        \item extrinsic: all descriptions (epistemic) and observations (alleoteric) are approximate and this error can compound quickly.
    \end{enumerate}

    Even very simple formulas containing arithmetic are undecidable. (Peano)

    No complete and consistent formal systems: we cannot construct a machine implementing the program of mathematics. (Godel)

    Determining equality between two mathematical formulas is undecidable. (Richardson)

    More generally, any nontrivial property is undecidable. (Rice)

    But there is hope! (Felleisen)

%    But can we organize mathematical knowledge as a decision tree? (Rich)

    Still, we can answer many questions exactly. If we restrict the reasoning system to incomplete queries, we can be consistient. In Section~\ref{sec:structural}, we will show some nice algorithms for doing so.

    Mathematics seems to be obsessed with completeness, much to its detriment. Completeness invades every definition. Definitions must hold over all cases, including the adversarial ones. This is good, because it encourages rigor in a highly adversarial field. But we must not confuse rigor with truth.

    The truth is, most people prefer useful abstractions to complete specifications. Most people just want to write dependable software.

    To be more inclusive of our audience, we should optimize our definitions for consistency, at the expense of completeness.

    \subsection{Rewriting approaches}

    We can rewrite stuff using rewrite systems.

    Knuth-Bendix completion is a semialgorithm...

    Suppose we have two functions $f_1: \mathcal{I} \rightarrow \mathcal{O}$ and $f_2: \mathcal{I}\rightarrow \mathcal{O}$.


    \subsection{Testing approaches}

    We can test using property-based / metamorphic testing. Some strategies:

    Isomorphism testing
    Metamorphic testing

    \subsection{Similarity based approaches}

    How do we know when we are approaching equality? We can use a distance metric.

    TODO: Define probability distributions, integration, kernel functions and metrics.

    We want a metric $M_\theta: (\mathcal{I}\rightarrow{O}) \times (\mathcal{I}\rightarrow{O})\rightarrow \mathbb{R}$ between two functions, which predicts their semantic similarity. In other words, the closer two functions are with respect to $M_\theta$, the more likely they are to be equal. During test time, we query a dataset for the $k$ most similar functions, and try to unify them extensionally and intensionally. We can train a metric $M_\theta$ and discriminator $D_\theta$ on pairs of random functions $f_1$ and $f_2$, to predict their similarity. During inference, we let the discriminator sample random inputs $\hat i_1 \ldots n \sim D_\theta(\hat i \mid f_1, f_2)$ from its latent distribution, conditioning on the structure of $f_1$ and $f_2$. In other words, we want a model that predicts similarity and outputs values which are likely to demonstrate instances of inequality.

    \section{Developing Software}

    We want to search through a software knowledge base for an error and stack trace, then use the information in the KB to repair our bug.

    \begin{enumerate}
        \item Efficiently searching corpus for a pattern
        \item Identifying alignment and matching results
        \item Incorporating information into user's context
    \end{enumerate}


    \subsection{Code search}

    We can search for code using various strategies...

    \subsection{Knowledge alignment}

    Knowledge alignment, entity matching, ...

    \subsection{Graph rewriting}

    \subsection{Graph edit distance}

    model fragments of code and natural language as a graphs and learn a distance metric which captures the notion of similarity. Some graphs will be incomplete, or missing some features, others will have extra information that is unnecessary.

    Given a piece of code and the surrounding context (e.g. in an IDE or compiler), search a database for the most similar graphs, then to recommend them to the user (e.g. fixes or repairs for compiler error messages), or suggest some relevant examples to help the user write some incomplete piece of code.It is similar to a string edit distance, but for graph structured objects. There are a few pieces to this:

    \begin{enumerate}
    \item Semantic segmentation (what granularity to slice?)
    \item Graph matching (how to measure similarity?)
    \item Graph search (how to search efficiently?)
    \item Recommendation (how to integrate into user's code)
    \end{enumerate}

    The rewriting mechanism is similar to a string edit distance, but for graphs. One way of measuring distance could be measuring the shortest number of steps for rewriting a graph A to graph B, ie. the more "similar" these two graphs are the fewer rewriting steps it should take.

    \subsection{Future work}

    Semirings arise in strange and marvelous places. (min, +), (max, \times)

    \begin{enumerate}
    \item https://people.cs.kuleuven.be/~luc.deraedt/Francqui4ab.pdf#page=71
    \item http://www.mit.edu/~kepner/GraphBLAS/GraphBLAS-Math-release.pdf#page=11
    \end{enumerate}

    \bibliography{exam_proposal}
    \bibliographystyle{plain}
\end{document}