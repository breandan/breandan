%! suppress = Unicode
%! Author = breandan
%! Date = 11/16/20

% Preamble
\documentclass[11pt]{article}

% Packages
\usepackage{amsmath}
\usepackage[pdf]{graphviz}
\usepackage{amssymb}
\usepackage{mathrsfs}

\usepackage{unicode-math}
\DeclareMathAlphabet{\mathcal}{OMS}{cmsy}{m}{n}
\usepackage{cancel}
\newcommand{\nDownarrow}{\ensuremath{\text{ }\cancel{\Downarrow}\text{ }}}
\usepackage{centernot}

\usepackage{tikz-cd}
\usepackage{amsfonts}
%\usepackage{prooftrees}
\usepackage{bussproofs}
\usepackage{hyperref}
\renewcommand{\sectionautorefname}{\S}
\renewcommand{\subsectionautorefname}{\S}
\usepackage{natbib}
\usepackage{float}
\usepackage{xcolor}


\title{Pattern Recognition in Procedural Knowledge}
\author{Breandan Considine}
\date{\today}

% Document
\begin{document}
    \maketitle

    \tableofcontents
    \pagebreak


    \section{Introduction}

    Historically, most knowledge was stored as natural language. A growing portion is now \textit{code}~\citep{allamanis2018survey}. Code is procedural knowledge intended for execution by a machine. Though it shares many statistical properties in common with natural language~\citep{hindle2012naturalness}, code is written in a formal language with a deterministic grammar and denotational semantics~\citep{pierce2010software}. We can use this specification to precisely reason about operational or procedural correctness.

    Prior work explored differentiable programming~\citep{considine2019programming}. Differentiability plays a key role in learning, but does not provide the necessary vocabulary to describe human knowledge. In order to capture human knowledge and begin to reason as humans do, programs must be able to express the concept of \textit{uncertainty}. In this work, we propose a set of tools and techniques for reasoning about uncertainty in the form of procedural knowledge.

    To reason about procedural knowledge, we must first define what it means for two procedures to be equal. Although equality is known to be undecidable in most languages, various equivalence tests and semi-decision procedures have been developed. For example, we could rewrite said procedures~\citep{baader1999term}, compare them in various contexts~\citep{felleisen1990expressive}, and simulate or execute them on various input data~\citep{chen2020metamorphic} so as to ascertain their exact relationship.

    In practice, exact equality is too rigid to operationalize. A more useful theory would allow us to compare two procedures in the presence of naturally-arising stochasticity. What is the probability of observing local variations? How are those observations related? And how do local variations affect global behavior? In order to correctly pose these questions and begin to answer them, we must be able to reason probabilistically.

    Graphs are a natural representation for both procedural knowledge~\citep{allamanis2017learning} and probabilistic reasoning~\citep{pearl2014probabilistic}. The language of linear algebra provides a unifying framework for many graph algorithms and program analysis tasks~\citep{kepner2011graph}. Recent evidence suggests probabilistic inference is tractable for a large class of graphical models~\citep{choi2020probabilistic}. And sparse matrix representations enable efficient processing of large graphs on modern graphics processors~\citep{kepner2016mathematical}.

    In this work, we will define exact and approximate equality and cover some deterministic and probabilistic algorithms for deciding it. We will then describe a few graph representations for encoding approximate procedural knowledge. Finally, we will discuss some opportunities for applying these ideas to search-based software engineering, in particular code search, vulnerability detection, fault localization and program repair.


    \section{From exact to approximate equality}\label{sec:definitions}

    Reason is the source of all human knowledge. In order to understand reason, we need to understand how concepts are related. Next to identity, one of the simplest relations between concepts is equality.

    For such an important concept, the notation for equality is recklessly overloaded in mathematics and computer science. For example, the expression $x = y$ may denote: (1)~define $x$ to be $y$, (2)~$x$ and $y$ are the same, (3)~are $x$ and $y$ the same? (4)~$x$ and $y$ are exchangeable, (5)~assign $y$ to $x$, (6)~assign $x$ to $y$, among other peculiar programming idioms. If two expressions are equal, it is generally possible to treat them in the same manner.

    But this convention does not always hold! Suppose we need to compute the derivative of a logical function with respect to its inputs. The trouble is, logical equality is not differentiable. Consider the Kronecker $δ$-function: %$δ_k: \mathbb{N}^2\rightarrow \mathbb{B}$:

    $$
    δ_k(x, y) :=
    \begin{cases}
        1 \text{ if } x \overset{?}{=} y, \\
        0 \text{ otherwise }\\
    \end{cases}
    $$

    When encountering $δ_k$, how should we represent its derivative? Since $\mathbb{B}$ is finite, $δ_d^{-1}(B\subset \mathbb{B})$ is not open, thus $δ_d$ is not continuous and $\nablaδ_k$ is undefined. Now consider the Dirac $δ$-function, which is defined as follows: %$δ_d: \mathbb{R}^2 \rightarrow \mathbb{R}$,

    $$
    \forall f \in \mathbb{R}^2 \rightarrow \mathbb{R}, \int_{\mathbb{R}^2} f(x,y)δ(x-a,y-b)d(x, y) \overset{Δ}{=} f(a,b)
    $$

    Unlike $\nablaδ_k$, it can be shown that $\nablaδ_d$ is well-behaved everywhere on $\mathbb{R}^2$. However we encounter an important distinction between intensional and extensional equality. Unlike elementary functions, there exist many functions which can only be described indirectly, e.g. a probability distribution on a set of measure zero. Nevertheless, these constructions are convenient abstractions for modeling many physical and computational processes.

    Neither $\nablaδ_k$ nor $δ_d$ are a satisfactory basis for equality. To allow a more flexible definition of the $=$ operator, we require a relation which approximates the logical properties of $δ_k$, but can be made differentiable like $δ_d$. A more general notion is the concept of an \textit{equivalence relation}. An equivalence relation $\equiv$ is a binary relation with the following logical properties:

    \begin{prooftree}
        \bottomAlignProof
        \AxiomC{}
        \UnaryInfC{$a \equiv a$}
        \noLine
        \UnaryInfC{}
        \noLine
        \UnaryInfC{\textit{Identity}}
        \DisplayProof
        \hskip 1.5em
        \bottomAlignProof
        \AxiomC{$a \equiv b$}
        \UnaryInfC{$b \equiv a$}
        \noLine
        \UnaryInfC{}
        \noLine
        \UnaryInfC{\textit{Symmetry}}
        \DisplayProof
        \hskip 1.5em
        \bottomAlignProof
        \AxiomC{$a \equiv b$}
        \AxiomC{$b \equiv c$}
        \BinaryInfC{$a \equiv c$}
        \noLine
        \UnaryInfC{}
        \noLine
        \UnaryInfC{\textit{Transitivity}}
        \DisplayProof
        \hskip 1.5em
        \bottomAlignProof
        \AxiomC{$a \equiv b$}
        \UnaryInfC{$f(a) \equiv f(b)$}
        \noLine
        \UnaryInfC{}
        \noLine
        \UnaryInfC{\textit{Congruence}}
    \end{prooftree}

    \pagebreak\subsection{Decidability}\label{sec:algorithms}

    To determine whether two expressions are equal, we need a decision procedure. Various approaches for deciding exact and approximate equality in the deterministic and probabilistic setting are possible. We list a few below.

    \begin{table}[H]
        \centering
        \begin{tabular}{c|l|l|}
            \cline{2-3} & \textbf{Deterministic} & \textbf{Probabilistic} \\ \hline
            \multicolumn{1}{|c|}{\textbf{Exact}} & \begin{tabular}[c]{@{}l@{}}
                                                       Type Checking\\ Model Checking
            \end{tabular} & \begin{tabular}[c]{@{}l@{}}
                                Variable Elimination\\Probabilistic Circuits
            \end{tabular} \\ \hline
            \multicolumn{1}{|c|}{\textbf{Approximate}} & \begin{tabular}[c]{@{}l@{}}
                                                             Software Testing\\Dynamic Analysis
            \end{tabular} & \begin{tabular}[c]{@{}l@{}}
                                Monte Carlo Methods\\Bayesian Networks
            \end{tabular} \\ \hline
        \end{tabular}
    \end{table}

    It is seldom the case that two semantically equal expressions are trivially equal: we must first perform some computation to establish their equality. In the exact setting, this procedure might be summarized as follows:

    \begin{enumerate}
        \item Rewrite: Either enumerate a set of equivalent expressions, or reduce the proposition into normal form if possible, then,
        \item Compare: Perform a computationally trivial (e.g. $\mathcal{O}(n)$) comparison.
    \end{enumerate}

    Unfortunately, exact equality is known to be undecidable in first~\cite{godel1929vollstandigkeit} and higher order theories~\cite{godel1931formal}. We know there can be no machine which accepts every equality and rejects every disequality in a universal language~\cite{turing1937computable}. By extension, any nontrivial property of partial functions is undecidable~\cite{rice1953classes}. %Equality between two elementary mathematical functions is undecidable~\cite{richardson1994identity}.

    %Is there any left hope? Yes! (Felleisen)

    Tractability may be related to, but is not contingent upon decidability. When decidable, equality may be intractable in practice, and languages where equality is undecidable may have decidable fragments. But even when exact equality is intractable, we may be able to construct a probabilistic decision procedure (PDP) or semidecision procedure (SDP) terminating for all practical purposes. The latter approaches fall into two broad categories:

    \begin{itemize}
        \item Execute: Evaluate the program by running it on a small set of inputs
        \item Sample: Build a probabilistic model and sample from its distribution
    \end{itemize}

    In the following section, we will introduce a few compatible theories corresponding to intensional and extensional equality, then build on those definitions to include recent approaches to exact and approximate equality in the determinsitic and probabilistic setting. In so doing, we will see there is a delicate tradeoff between complexity, sensitivity and specificity.

%    Mathematics is often a useful approximation of reality, but many mathematical concepts are impossible to realize. This is due to two problems:
%
%    \begin{enumerate}
%        \item intrinsic: the mechanics of the system are intrinsically unsound or impossible to mechanize. How do we encode $\lim_{x \rightarrow \infty}$?
%        \item extrinsic: all descriptions (epistemic) and observations (alleoteric) are approximate and this error can compound quickly.
%    \end{enumerate}


%    But can we organize mathematical knowledge as a decision tree? (Rich)

%    Still, we can answer many questions exactly. If we restrict the reasoning system to incomplete queries, we can be consistent.

    \subsection{Intensional equivalence}\label{subsec:in-eq}

    Let $Ω \subseteq \mathcal{F} \times \mathcal{F}$ be a relation on representable functions which are closed under composition. We say two representations $f, g \in \mathcal{X} \rightarrow \mathcal{Y}$ are intensionally equal under $Ω$ if we can establish that $g \in Ω^n(f)$ for some $n \in \mathbb{N}$.

    \begin{prooftree}
        \AxiomC{$f, g: \mathcal{X \rightarrow Y} \in Γ^0_{g}$}
        \RightLabel{\textsc{Init}}
        \UnaryInfC{$Γ^0_{g} \vdash \{f\}, \{(f, f)\}$}
%        \UnaryInfC{\textit{Commutivity}}
        \DisplayProof
        \hskip 1em
        \AxiomC{$Γ^{n}_{g} \vdash E \subseteq \mathcal{F}, G \subseteq E \times E$}
        \RightLabel{\textsc{Sub}}
        \UnaryInfC{$Γ^{n+1}_{g} \vdash \bigcup\limits_{\substack{e \in E\\\sigma \in Ω}}e'\leftarrow e[\sigma_1\rightarrow\sigma_2], (e, e')$}
        \DisplayProof
        \vskip 1em
        \AxiomC{$Γ^n_{g} \vdash E, G$}
        \AxiomC{$g \in E$}
        \RightLabel{\textsc{Eq}}
        \BinaryInfC{$Γ^n_{g} \vdash f \equiv_Ω g$ by $G^{-n}(g)$}
        \DisplayProof
        \hskip 1em
        \AxiomC{$Γ^n_{g} \vdash E$}
        \AxiomC{$Γ^{n+1}_{g}\vdash E$}
        \AxiomC{$g\notin E$}
        \RightLabel{\textsc{Neq}}
        \TrinaryInfC{$Γ^{n+1}_{g} \vdash f \not\equiv_Ω g$}
    \end{prooftree}

    \noindent For example, suppose we are given $f: \{a, b, c\} \mapsto a b c, g: \{a, b, c\} \mapsto c b a$ and $Ω := \{(a, a), (ab, ba)\}$. Indeed, $\textsc{Eq}[f, g]$ can be established as follows:

    \vspace{-10pt}\begin{prooftree}
        \def\fCenter{\ := \ }
        \Axiom$f \fCenter abc, \text{ } g := cba \in Γ_g^0$
        \def\fCenter{\ \vdash\ }
        \RightLabel{\textsc{Init}}
        \UnaryInf$Γ^0_{g} \fCenter \{abc\}, \{(abc, abc)\}$
        \RightLabel{\textsc{Sub}}
        \UnaryInf$Γ^1_{g} \fCenter \{\ldots, bac, acb\}, \{\ldots, (abc, bac), (abc, acb)\}$
        \RightLabel{\textsc{Sub}}
        \UnaryInf$Γ^2_{g} \fCenter \{\ldots, bca, cab\}, \{\ldots, (bac, bac), (acb, acb), (bac, bca), (acb, cab)\}$
        \RightLabel{\textsc{Sub}}
        \UnaryInf$Γ^3_{g} \fCenter \{\ldots, \mathbf{cba}\}, \{\ldots, (bca, bca), (cab, cab), (cab,\textbf{cba})\}$
        \RightLabel{\textsc{Eq}}
        \UnaryInf$Γ^3_{g} \fCenter f \equiv_Ω g\text{ by } G^{-3}(g:=cba) = \{f := abc\}$
    \end{prooftree}

    \noindent We can visualize $G$ as a directed graph, omitting all loops. Notice how each path converges to the same term, a property known as $\textit{strong confluence}$.

    \hspace{12pt}\digraph[scale=0.5]{abcint}{
    node[ fontname="CMU Classical Serif" fontsize=20 ];
    edge[ fontname="CMU Classical Serif" fontsize=18 ];
    rankdir=LR;
    len=3;
    node [shape=Mrecord];

    a [ label="abc"; ]
    b [ label="bac"; ]
    c [ label="acb"; ]
    d [ label="bca"; ]
    e [ label="cab"; ]
    f [ label="cba"; ]

%    a -> a [label="Σ₀"]
    a -> b [label=<Ω<SUB>1</SUB>> minlen=2]
    a -> c [label=<Ω<SUB>1</SUB>> minlen=2]
    b -> d [label=<Ω<SUB>1</SUB>> minlen=2]
    c -> e [label=<Ω<SUB>1</SUB>> minlen=2]
    e -> f [label=<Ω<SUB>1</SUB>> minlen=2]
    d -> f [label=<Ω<SUB>1</SUB>> minlen=2]
    }

    \noindent Let us suppose $|\mathcal{X}|, |Ω^*| \in \mathbb{N}$ and consider the complexity of establishing $\textsc{Eq}[f, g], \forall f \equiv_Ω g \in \mathcal{X} \rightarrow \mathcal{Y}$. It can be shown the above procedure requires:

    $$\mathcal{O}_{\textsc{Eq}} = \underset{i \leq n}{\text{max }}\underset{n}{\text{argmin}}\{|G| \mid Γ^i_{g} \vdash G, Γ_{g}^n = Γ_{g}^{n+1}\}$$ % \text{, or } \Theta_{\textsc{Eq}}(|\mathcal{X}|!) \text{ for } Ω.$$

    \noindent Assuming termination, $\mathcal{O}_{\textsc{Eq}} = \Theta_{\textsc{Neq}}$ although $\mathbb{E}[\Theta_\textsc{Eq}|f \equiv g]$ is more tractable. However termination is not necessarily guaranteed, e.g. $Ω' := \{(a, 1a)\}$. Equality and termination under arbitrary $Ω$ are known to be undecidable~\citep{baader1999term}.

    \subsection{Computational equivalence}\label{subsec:comp-eq}

    Clearly, the procedure defined in \S~\ref{subsec:in-eq} is highly sensitive to $|\mathcal{X}|$ and $Ω$. While equality may be tractable, disequality is definitely an obstacle. In the computational setting, we will see the opposite holds, ceteris paribus. %Two values $r, r': \mathcal{Y}$ share a normal form in which equality is trivial.

    \begin{prooftree}
        \AxiomC{$fg: \mathcal{X} \rightarrow \mathcal{Y}, Ω: \{\mathcal{X}\rightarrow \mathcal{Y}\}\rightarrow \mathcal{Y} \in Γ$}
        \RightLabel{\textsc{Inv}}
        \UnaryInfC{$Γ \vdash fg(Ω) \Downarrow f(Ω) \cdot g(Ω)$}
        \DisplayProof
        \hskip 1em
        \AxiomC{$Γ \vdash f$}
        \AxiomC{$Γ \vdash Ω$}
        \RightLabel{\textsc{Sub}}
%        \AxiomC{$i: \mathcal{X}$}
        \BinaryInfC{$Γ \vdash f(Ω) \Downarrow Ω[f]$}
        \DisplayProof
        \vskip 1em
        \AxiomC{$Γ, Γ' \vdash f(Ω) \Downarrow g(Ω) \text{ } \forall Ω$}
        \RightLabel{\textsc{Eq}}
        \UnaryInfC{$Γ, Γ' \vdash f \equiv g$}
        \DisplayProof
        \hskip 1em
        \AxiomC{$Γ, Γ' \vdash \exists Ω \mid f(Ω) \nDownarrow g(Ω) $}
        \RightLabel{\textsc{Neq}}
        \UnaryInfC{$Γ, Γ' \vdash f \not\equiv g$ by $Ω$}
    \end{prooftree}

    % big step semantics
%    Extensional equality asks``Do $f_1$ and $f_2$ behave in the same way over all inputs?''
    \noindent \textsc{Sub} loosely corresponds to $\eta$-reduction in the untyped $\lambda$-calculus, however $f \notin Ω$ is disallowed and we assume all variables are bound by \textsc{Inv}. Let us consider $f: \{a, b, c\}\mapsto abc, g: \{a, b, c\} \mapsto ac$ under $Ω:=\{(a, 1), (b, 2), (c, 2)\}$:

    %If we test $\hat i \in {-2, -1, 0, 1, \ldots}$, we have $f_1(-2)=f_2(-2)$, $f_1(-1)=f_2(-1)$, $f_1(0)=f_2(0)$, $f_1(1) \neq f_2(1)$. Once we detect an $f_1(\hat i) \neq f_2(\hat i)$, we can halt immediately.

    \vspace{-10pt}\begin{prooftree}
        \def\fCenter{\ :=}
        \def\defaultHypSeparation{\hskip -1.1in}
        \Axiom$f\fCenter abc, Ω:=\{(a, 1), (b, 2), (c, 2)\} \in Γ$
        \def\fCenter{\ \vdash\ }
        \RightLabel{\textsc{Inv}}
        \UnaryInf$Γ \fCenter a(Ω)\cdot bc(Ω)$
        \RightLabel{\textsc{Sub}}
        \UnaryInf$Γ \fCenter 1\cdot bc(Ω)$
        \RightLabel{\textsc{Inv}}
        \UnaryInf$Γ \fCenter 1\cdot b(Ω)\cdot c(Ω)$
        \RightLabel{\textsc{Sub}}
        \UnaryInf$Γ \fCenter 2\cdot c(Ω)$
        \RightLabel{\textsc{Sub}}
        \UnaryInf$Γ \fCenter f(Ω) \Downarrow 4$
        \def\fCenter{\ :=}
        \Axiom$g\fCenter ac, Ω:=\{(a, 1), (b, 2), (c, 2)\} \in Γ'$
        \def\fCenter{\ \vdash\ }
        \RightLabel{\textsc{Inv}}
        \UnaryInf$Γ' \fCenter a(Ω)\cdot c(Ω)$
        \RightLabel{\textsc{Sub}}
        \UnaryInf$Γ' \fCenter 1\cdot c(Ω)$
        \RightLabel{\textsc{Sub}}
        \UnaryInf$Γ' \fCenter g(Ω) \Downarrow 2$
        \RightLabel{\textsc{Neq}}
        \BinaryInfC{$Γ, Γ' \vdash f \not\equiv g$ by $Ω :=\{(a, 1), (b, 2), (c, 2)\}$}
    \end{prooftree}

    \noindent We can view the above process as acting on a dataflow graph, where \textsc{Inv} backpropagates $Ω$, and \textsc{Sub} returns concrete values $\mathcal{Y}$, here depicted on $g$:

    \hspace{-30pt}\digraph[scale=0.5]{abcext0}{
    node[ fontname="CMU Classical Serif" fontsize=20 ];
    edge[ fontname="CMU Classical Serif" fontsize=18 ];
    rankdir=LR;
    len=3;
    node [shape=Mrecord];

    f [ label="g"; ]
    a [ label="a"; ]
    c [ label="c"; ]
    d [ label="·"; ]

%    a -> a [label="Σ₀"]
    f -> d [label="Ω"]
    d -> a [label="Ω"]
    d -> c [label="Ω"]
    }\hspace{-20pt}\digraph[scale=0.5]{abcext1}{
    node[ fontname="CMU Classical Serif" fontsize=20 ];
    edge[ fontname="CMU Classical Serif" fontsize=18 ];
    rankdir=LR;
    len=3;
    node [shape=Mrecord];

    f [ label="g"; ]
    a [ label="a"; ]
    c [ label="c"; ]
    d [ label="·"; ]

%    a -> a [label="Σ₀"]
    d -> f [label="𝓡"]
    a -> d [label="𝓡"]
    c -> d [label="𝓡"]
    }

    \noindent Assuming $f, g \sim P(\mathcal{F}), Ω \overset{iid}{\sim} P_\textsc{Test}(Ω \mid f \not\equiv g)$ yields a fixed but unknown distribution, $P_\textsc{Neq}(Ω) = P(f(Ω) \nDownarrow g(Ω) \mid f \not\equiv g)$. Let $\Theta_\textsc{Inv} = 1$ for all $f, Ω$. The complexity of certifying $\textsc{Neq}$ in $n$ trials follows a geometric distribution:

    \vspace{-10pt}$$\Theta_\textsc{Neq} \sim (1 - P_\textsc{Neq}(Ω))^nP_\textsc{Neq}(Ω) \text{ with } \mathop{\mathbb{E}}[\Theta_\textsc{Neq}] = (1-P_\textsc{Neq}(Ω))P_\textsc{Neq}(Ω)^{-1}$$

    \noindent Although a single witness $Ω$ s.t. $f(Ω) \nDownarrow g(Ω)$ is sufficient for disequality, this procedure may be intractable depending on $|\mathcal{X}|, P_\textsc{Neq}(Ω)$ and $\Theta_{\textsc{Inv}}$. Other fuzzing methods for selecting $Ω$ based on the structure of $f$ are also possible.

%    We can test using property-based / metamorphic testing. Some strategies:

%    Isomorphism testing

%    Metamorphic testing


    \pagebreak\subsection{Observational Equivalence}

    As presented, both intensional (\S~\ref{subsec:in-eq}) and computational (\S~\ref{subsec:comp-eq}) equivalence require an external definition of equality to satisfy. One solution to this problem known as \textit{observational equivalence}~\cite{morris1969lambda} allows a language $\mathcal{L}$ to implement an internal mechanism to verify equality. Given $\mathcal{L}$, a term $t$, and one-hole context $C[\![\cdot]\!]$, our job is to check for termination: if $C[\![t]\!]$ is both well-defined and halts, we write $C[\![t]\!]\Downarrow$, otherwise $C[\![t]\!]\Uparrow$.

    \begin{prooftree}
        \AxiomC{$Γ \vdash C[\![t]\!]\Downarrow \iff C[\![t']\!]\Downarrow$ $\forall$ $C[\![\cdot]\!] \in \mathcal{L}$}
        \RightLabel{\textsc{Eq}}
        \UnaryInfC{$Γ \vdash t \equiv_{\mathcal{L}} t'$}
        \DisplayProof
        \vskip 1em
        \AxiomC{$Γ \vdash \exists C[\![\cdot]\!]\in \mathcal{L} \mid C[\![t]\!]\Uparrow $ and $ C[\![t']\!]\Downarrow$, or $C[\![t]\!]\Downarrow $ and $ C[\![t']\!]\Uparrow$}
        \RightLabel{\textsc{Neq}}
        \UnaryInfC{$Γ \vdash t \not\equiv_{\mathcal{L}} t'$ by $C[\![\cdot]\!]$}
    \end{prooftree}

%    http://www.cs.ox.ac.uk/people/samuel.staton/papers/fossacs-2019.pdf
%    http://users.ox.ac.uk/~scro3229/documents/birmingham-talk.pdf

    We can think of this definition as dual to computational equivalence: instead of searching for inputs which distinguish functions, we search for contexts which distinguish terms, or a proof that no such context exists. Two terms $t$ and $t'$ are contextually equivalent with respect to $\mathcal{L}$ if we can prove that for all contexts $C[\![\cdot]\!]$ in $\mathcal{L}$, $C[\![t]\!]$ halts if and only if $C[\![t']\!]$ halts -- if no such proof can be found, the test is inconclusive. While this definition does not admit a decision procedure, many promising SDPs exist.

%    $P(w_t = a | w_{t-1}, w_{t-2}\ldots, w_{t+1}, w_{t+2}, \ldots)\overset{?}{\approx} P(w_t = b | w_{t-1}, w_{t-2}\ldots, w_{t+1}, w_{t+2}, \ldots)$

%    We can define semantic equality in our setting using word embeddings.

    \subsection{Approximate Equivalence}\label{sec:ap-eq}

    Approximate equivalence requires a \textit{distance metric}, $δ: \mathcal{Z}\times\mathcal{Z}\rightarrow\mathbb{R}_{\geq 0}$. This is a generalized equivalence relation with the following logical properties:

    \begin{prooftree}
        \bottomAlignProof
        \AxiomC{$δ(a, b) = 0$}
        \UnaryInfC{$a \equiv_δ b$}
        \noLine
        \UnaryInfC{}
        \noLine
        \UnaryInfC{\textit{Definiteness}}
        \DisplayProof
        \hskip 1.5em
        \bottomAlignProof
        \AxiomC{$δ(a, b)$}
        \UnaryInfC{$δ(b, a)$}
        \noLine
        \UnaryInfC{}
        \noLine
        \UnaryInfC{\textit{Symmetry}}
        \DisplayProof
        \hskip 1.5em
        \bottomAlignProof
        \AxiomC{$a$}
        \AxiomC{$b$}
        \AxiomC{$c$}
        \TrinaryInfC{$δ(a, c) \leq δ(a, b) + δ(b, c)$}
        \noLine
        \UnaryInfC{}
        \noLine
        \UnaryInfC{\textit{Triangularity}}
    \end{prooftree}

    \noindent A \textit{kernel function} $Δ: (\mathcal{X}\rightarrow\mathcal{Y})^2\rightarrow \mathbb{R}_{\geq 0}$ can be defined as a metric on $\mathcal{Z}$ with some additional structure. Specifically for every kernel function, there exists a feature map $\varphi: (\mathcal{X}→\mathcal{Y}) → \mathcal{Z}$ such that $Δ: (f, g) \mapsto \left<\varphi(f), \varphi(g)\right>$. Given a feature map $\symbf\varphi: \mathbb{R}^m → \mathbb{R}^n$, constructing a kernel corresponds to finding $\Delta \in \mathbb{R}^{m \times m}$ symmetric PSD, i.e. $\Delta^\intercal = \Delta$, and $0 \leq \mathbf{x}^\intercal \Delta \mathbf{x}$ for all $\mathbf{x} \in \mathbb{R}^m$~\cite{mercer1909functions}. Consider the complexity of evaluating the inner product $\left<\symbf\varphi(f), \symbf\varphi(g)\right>$ in feature space. The computational benefit of a kernel becomes apparent when $m \ll n$. Rather than applying $\symbf\varphi$, then evaluating $\left<f^\intercal\symbf\varphi, g^\intercal\symbf\varphi\right>$, we may simply apply $\Delta(f, g)$ for an $\Theta(m^2 - n^2)$ speedup. Known as the \textit{kernel trick}, this shortcut may be easier to visualize categorically, where $f, g: \mathcal{X}→\mathcal{Y}$.

    \[\begin{tikzcd}
          (\mathcal{X→\mathcal{Y}})^2 \arrow{r}{\varphi} \arrow[labels=below left]{dr}{\Delta} & \mathcal{Z}\times\mathcal{Z} \arrow{d}{\left<\cdot, \cdot\right>} \\
          & \mathbb{R}_{\geq 0}
    \end{tikzcd}
    \]

    \noindent By planting a single valid kernel $\Delta$, one can grow a tree of kernel functions $▲ \subset (\mathcal{X}→\mathcal{Y})^2 → \mathbb{R}_{\geq 0}$, forming a grammar with the following productions:

    % https://papers.nips.cc/paper/2000/file/4e87337f366f72daa424dae11df0538c-Paper.pdf
    % https://www.stat.berkeley.edu/~bartlett/courses/2014spring-cs281bstat241b/lectures/20-notes.pdf#page=17
    % https://people.eecs.berkeley.edu/~jordan/kernels/0521813972c03_p47-84.pdf#page=29
%    http://www.cs.cmu.edu/~aarti/Class/10701_Spring14/slides/kernel_methods.pdf#page=38

    \begin{prooftree}
        \bottomAlignProof
        \AxiomC{$Δ \in ▲, k \in \mathbb{R}_{> 0}$}
        \UnaryInfC{$kΔ \in ▲$}
        \DisplayProof
        \hskip 0.6em
        \bottomAlignProof
        \AxiomC{$Δ_{1, 2} \in ▲$}
        \UnaryInfC{$Δ_1 + Δ_2 \in ▲$}
        \DisplayProof
        \hskip 0.6em
        \bottomAlignProof
        \AxiomC{$Δ_{1, 2} \in ▲$}
        \UnaryInfC{$Δ_1Δ_2 \in ▲$}
        \DisplayProof
        \hskip 0.6em
        \bottomAlignProof
        \AxiomC{$Δ \in ▲, f \in (^*→\mathcal{Z})^2$}
        \UnaryInfC{$Δ\circ f \in ▲$}
    \end{prooftree}


    \noindent Some common elementary kernel functions are listed below:

    %, we can construct the corresponding kernel as follows...

    \begin{center}
        \begin{tabular}{|c|c|c|}
            \hline
            Polynomial & $k(\mathbf{f},\mathbf{g}):=(\mathbf{f}^{T}\mathbf{g}+r)^{n}$ & $\quad \mathbf{f},\mathbf{g}\in \mathbb {R} ^{d}, n \in \mathbb N, r\geq 0$ \\ \hline
            Laplacian & $k(\mathbf {f}, \mathbf {g}):=\exp \left(-{\frac {\|\mathbf {f} -\mathbf {g} \|}{\sigma}}\right)$ & $\quad \mathbf{f},\mathbf{g}\in \mathbb {R} ^{d},\sigma >0$ \\ \hline
            Gaussian RBF & $k(\mathbf {f}, \mathbf {g}):=\exp \left(-{\frac {\|\mathbf {f} -\mathbf {g} \|^{2}}{2\sigma ^{2}}}\right)$ & $\quad \mathbf{f},\mathbf{g}\in \mathbb {R} ^{d},\sigma >0$ \\ \hline
            String Kernel & ... & ... \\ \hline
            Tree Kernel & ... & ... \\ \hline
            Graph Kernel & ... & ... \\ \hline
        \end{tabular}
    \end{center}

    We want a kernel function $M_\theta: (\mathcal{X} → \mathcal{Y}) \times (\mathcal{X}→\mathcal{Y})→\mathbb{R}$ between two computable functions, which predicts their semantic similarity. In other words, the closer two functions are with respect to $M_\theta$, the more likely they are to be equal...

%    It is often the case we are given a dataset $P_{\text{Train}}, P_{\text{Test}}: (X \times Y)^n$ sampled from an inaccessible distribution $P^{(1..n)} \overset{iid}{\sim} P_\text{Gen}(X, Y)$, and want to compare a deterministic computer program $\hat p: X \times \Theta → Y$ approximating $P_\text{Gen}$ to detect errors with respect to that dataset. One mechanism for doing so can be described as follows:...

%    \begin{prooftree}
%        \AxiomC{$Γ \vdash f \circ g, f: Z → Y, g: X → Z, B: := P_\text{Train}[i]$}
%        \RightLabel{\textsc{SGD}}
%        \UnaryInfC{$Γ \vdash t \equiv_{\mathcal{L}} t'$}
%        \DisplayProof
%        \vskip 1em
%%        \AxiomC{$Γ \vdash \exists C[\![\cdot]\!]\in \mathcal{L} \mid C[\![t]\!]\Uparrow $ and $ C[\![t']\!]\Downarrow$, or $C[\![t]\!]\Downarrow $ and $ C[\![t']\!]\Uparrow$}
%%        \RightLabel{\textsc{ERM}}
%%        \UnaryInfC{$Γ \vdash t \not\equiv_{\mathcal{L}} t'$ by $C[\![\cdot]\!]$}
%    \end{prooftree}

    %Given an oracle $M: \mathcal I → \mathcal R$, and a set of inputs $X := \{x^{(i)}\}_{i=1}^n$ and outputs $Y := \{y^{(i)} := M(x^{(i)})\}_{i=1}^n$, a metamorphic relation (MR) is a relation $M \subset \mathcal I^z \times \mathcal R^z$ where $z \geq 2$. In the simplest case, an MR is an equivalence relation $\mathcal R$, i.e.: $\langle \mathbf x, \mathbf y, \mathbf x', \mathbf y' \rangle \in \mathcal R \Leftrightarrow \mathbf x \sim_{\mathcal R} \mathbf x' \Leftrightarrow \mathbf P(\mathbf x) \approx \mathbf P(\mathbf x')$.

    \pagebreak\subsection{Probabilistic Equivalence}\label{sec:pr-eq}

    Though precise, the prior definitions of equality are far too rigid in practice. The spaces involved either lack formal semantics or are intractable to exhaustive search. We now turn to a language for probabilistic reasoning, which admits among other things, a decision procedure for probabilistic equivalence.

%    How do we know when we are approaching equality? We can use a distance metric.

    % https://towardsdatascience.com/beyond-weisfeiler-lehman-approximate-isomorphisms-and-metric-embeddings-f7b816b75751

    Various measures have been proposed...

    Hypothesis testing

    Kolmogorov-Smirnov

    Kantorovich-Rubinstein

    Kullback-Leibler

    L\'evy-Prokhorov

    Gromov-Hausdorff

    Jensen-Shannon

    Cauchy-Schwartz

    EMD


    TODO: Define probability distributions, integration, kernel functions and metrics.

%    During test time, we query a dataset for the $k$ most similar functions, and try to unify them computationally and intensionally. We can train a metric $M_\theta$ and discriminator $D_\theta$ on pairs of random functions $f_1$ and $f_2$, to predict their similarity. During inference, we let the discriminator sample random inputs $\hat i_1 \ldots n \sim D_\theta(\hat i \mid f_1, f_2)$ from its latent distribution, conditioning on the structure of $f_1$ and $f_2$. In other words, we want a model that predicts similarity and outputs values which are likely to demonstrate instances of inequality.

    \pagebreak

    \section{From computation to knowledge graphs}\label{sec:graphs}

    % https://en.wikipedia.org/wiki/Duality_(optimization)

    Duality is an important concept in mathematical optimization. Many optimization problems can be seen as dual to each other: KKT and SVM duality. Duality occurs in automatic differentiation with dual number arithmetic.

    Duality is also an important concept in computer science. One famous example is the duality between code and data: in \textit{homoiconic} languages, we can treat code as data and data as code. cf. Kleene's recursion theorem.

    One way to view automatic differentiation is that it allows us to compute the sensitivity of numerical values in a fixed computation graph. What we wanted to compute sensitivities with respect to changes in the computation graph itself? For that, we need to define a the graph as an algebraic object.

    \subsection{Algebraic graphs}\label{subsec:algebraic-graphs}

    Graphs can be modeled algebraically~\citep{weisfeiler1968reduction}  using algebraic data types \citep{mokhov2017algebraic}.

    Graphs are algebraic structures...

    Semirings arise in strange and marvelous places. $(min, +), (max, \times)$

    \begin{enumerate}
        \item \url{https://people.cs.kuleuven.be/~luc.deraedt/Francqui4ab.pdf#page=71}
        \item \url{http://www.mit.edu/~kepner/GraphBLAS/GraphBLAS-Math-release.pdf#page=11}
    \end{enumerate}

    \subsection{Programs are graphs}\label{sec:program-graphs}

    computation graphs~\citep{breuleux2017automatic} in machine learning

    e-Graphs~\citep{willsey2020egg} in reasoning systems

    arithmetic circuits~\citep{miller1988efficient} in numerical computing

    probabilistic circuits~\citep{choi2020probabilistic} in probabilistic modeling community

    \subsection{Probabilistic graphical models}\label{sec:pgms}

    Probabilistic graphical models (PGMs) are very expressive, but even approximate inference on belief networks (BNs) is NP-hard~\citep{dagum1993approximating} We can faithfully represent a large class of PGMs and their corresponding distributions as probabilistic circuits (PCs)~\citep{choi2020probabilistic}, which are capable of exact inference in polynomial time and empirically tractable to calibrate using SGD or EM. PCs share many algebraic properties with PGMs and can propagate statistical estimators like variance and higher moments using simple rules.

    \subsection{Knowledge graphs}

    Knowledge graphs~\citep{hogan2020knowledge} are multi-relational graphs whose nodes and edges possess a type. Two entities can be related by multiple types, and each type can relate many pairs of entities. We can index an entity based on its type for knowledge retrieval, and use types to reason about compound queries, e.g. ``Which company has a direct flight from a port city to a capital city?''

%    \subsection{Message passing algorithms}
%
%    Many algorithms can be implemented as message passing on graphs...

    \pagebreak


    \section{From procedural knowledge to code}\label{sec:applications}

    Experts typically encode their knowledge using pen and paper and leave developers to decipher it. It is somewhat tedious, but generally possible for skilled programmers to translate textual information into computer programs. Unfortunately, there are many equivalent ways to translate text into code -- the same algorithm implemented in the same language by different authors is seldom written the same way. Usually we end up reinventing the wheel. So we need some mechanism to detect exact or approximate equality in procedural knowledge.

    Instead of the Sisyphean task of forever translating these ideas from scratch, coders need to step back and think: Is it possible to just encode the axioms and enough knowledge to derive a family of algorithms, then let the compiler derive the most appropriate procedure for computing some desired quantity? A good compiler might be able to use those facts to optimize computation graphs, e.g. for latency or numerical stability. This has been the holy grail of declarative programming.

    Short of that, can we have some kind of \textit{bibliotheca universalis} containing many human-written code examples, which a human could select from manually (a la Hoogle~\citep{james2020digging}) -- or even better -- which could be linked to during compilation. We can think of big code as a kind of procedural knowledge system, describing common data transformations. We would like a way to extract common snippets and reason about those transformations, for example to detect similar procedures or optimize an existing procedure.

    Knowledge systems or \textit{ontologies} are a collection of related facts which have been established by human beings. For example, we can treat mathematics as a knowledge base of rewrite rules. This has been successfully operationalized in Theano~\citep{bergstra2010theano}, Kotlin$\nabla$~\citep{considine2019kotlingrad} and other DSLs. More broadly, we can also think of constructive mathematics libraries like Metamath~\citep{megill2006metamath}, Rubi~\citep{rich2009knowledge}, Probonto~\citep{swat2016probonto} and KMath~\citep{nozik2019kotlin} as working towards this same goal.

    In knowledge graphs, approximate equality is known as entity \textit{alignment} or \textit{matching}. With a probabilistic matching algorithm, we could accurately detect near duplicates in a codebase. We could retrieve code samples to assist developers writing unfamiliar code. And we could search for bugs in code or fixes from a knowledge base to repair them. Probabilistic reasoning can be gainfully employed on these and many related tasks.

    Suppose we want to search through a software knowledge base for an error and stack trace, then use the information in the KB to repair our bug.

    \begin{enumerate}
        \item Efficiently searching corpus for a pattern
        \item Identifying alignment and matching results
        \item Incorporating information into user's context
    \end{enumerate}

    \subsection{Code search}

    model fragments of code and natural language as a graphs and learn a distance metric which captures the notion of similarity. Some graphs will be incomplete, or missing some features, others will have extra information that is unnecessary.

    Given a piece of code and the surrounding context (e.g. in an IDE or compiler), search a database for the most similar graphs, then to recommend them to the user (e.g. fixes or repairs for compiler error messages), or suggest some relevant examples to help the user write some incomplete piece of code.It is similar to a string edit distance, but for graph structured objects. There are a few pieces to this:

    \begin{enumerate}
        \item Semantic segmentation (what granularity to slice?)
        \item Graph matching (how to measure similarity?)
        \item Graph search (how to search efficiently?)
        \item Recommendation (how to integrate into user's code)
    \end{enumerate}

    The rewriting mechanism is similar to a string edit distance, but for graphs. One way of measuring distance could be measuring the shortest number of steps for rewriting a graph A to graph B, i.e. the more "similar" these two graphs are the fewer rewriting steps it should take.

    \subsection{Fault localization}

    searching for stack trace on stackoverflow

    \subsection{Program repair}

    adapt some knowledge to match user's context

    \subsection{eDSL generation}

    Take a procedural knowledge base, generate a DSL from it.

    \pagebreak
    \bibliography{exam_proposal}
    \bibliographystyle{plain}
\end{document}