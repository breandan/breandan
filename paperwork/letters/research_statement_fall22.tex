%! Author = breandan
%! Date = 7/17/22

% Preamble
\documentclass[11pt]{article}

% Packages
\usepackage{amsmath}
\usepackage[table]{xcolor}
\usepackage{array}

\newcolumntype{P}[1]{>{\raggedright\vrule height4ex width 0pt}p{#1}<{\vrule depth 2.5ex width 0pt}}

\usepackage{pgfgantt}
\ganttset{
    y unit title=0.5cm,
    y unit chart=1cm,
    x unit =0.75cm,
    vgrid,hgrid,
    title height=1,
    title/.style={fill=none},
    title label font=\footnotesize,
    bar/.style={fill=gray},
    bar height=1,
    bar top shift=0,
    progress label text={},
    group right shift=0,
    group height=.6,
    group peaks width={0.2},
    inline,
    bar label node/.style={text width=3.5cm,
    align=right,
    anchor=east,
    font=\small}
}

\begin{document}
    \title{Research Statement}
    \author{Breandan Considine}
    \maketitle
    Our research investigates how to make programming more accessible, safe and user-friendly. I am espeically interested in studying (1) how code is written in today's programming languages (2) how to make the programming languages of tomorrow more safe and ergonomic (3) how human intuition and type systems can work together more harmoniously in the future using type theory, SAT solving, and neural program synthesis. In particular:\\

    \begin{itemize}
        \item I am interested in helping programmers become more productive by providing more expressive language constructs, more attentive tools, and more interactive development environments. For example, I have created an \emph{IDE plugin} that allows programmers to search for partially completed code fragments consistient with a context-free grammar.
        \item I am also interested in what we can learn by observing programmers writing code in their natural habitat, and how programming can be made more user-friendly by anticipating and intervening on commonly occurring mistakes. For example, I am investigating how to repair simple programming errors using an \textit{error correcting parser}.
    \end{itemize}

    \noindent How are programming languages used in practice? \\

    \begin{itemize}
        \item I am interested in how people actually use programming languages and tools in their everyday work. For example, I am studying how software developers use \emph{code search engines} and \emph{code completion} tools.
        \item I am also interested in studying how large language models behave under variance. I want to measure their \textit{compositional generalization} by evaluating their performance on real world programming scenarios.
    \end{itemize}

    I believe that during the process of writing code, the computer should be actively thinking (like a pair programmer) and searching for paths through edit space. Each keystroke provides slightly more context about the intended goal. In order to produce delightful user experiences, it must be constantly thinking, ``How can I help to carry the author's cognitive burden?''\\

    \noindent How can we scale up automated programming and reasoning using GPUs?\\

    \noindent Can we design a brand new programming language in which everything, including type checking, compilation and execution is just parsing?\\

    TODO: Ongoing research directions (ML4Code, Arrays, Parsing, DevTools) \\

    TODO: Future work (LLM evaluation, Term Rewriting, Graph computation) \\

    TODO: Research Agenda (ProbProg, PLDI, SPLASH, learning to code) \\

    \noindent I plan to devote the upcoming year towards the following research activities: \\

    \noindent\hspace{-12pt}\begin{ganttchart}{1}{10}
        \gantttitle{2022}{5}
        \gantttitle{2023}{7} \\

        \gantttitle{Aug}{1} \gantttitle{Sep}{1} \gantttitle{Oct}{1} \gantttitle{Nov}{1} \gantttitle{Dec}{1}
        \gantttitle{Jan}{1} \gantttitle{Feb}{1} \gantttitle{Mar}{1} \gantttitle{Apr}{1} \gantttitle{May}{1}\\

        \ganttbar[progress=100,inline=false] {LIVE Paper} {1}{1}\\
        \ganttbar[progress=100,inline=false] {PLDI Submission} {2}{4}\\
        \ganttbar[progress=100,inline=false] {SPLASH/OOPSLA} {5}{5} \\
        \ganttbar[progress=100,inline=false] {NeurIPS Submission} {6}{10}
    \end{ganttchart}
\end{document}