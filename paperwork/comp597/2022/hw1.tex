%! Author = breandanconsidine
%! Date = 9/19/21

% Preamble
\documentclass[11pt]{article}

% Packages
\usepackage{amsmath}
\usepackage{hyperref}
\usepackage{amsfonts}
\usepackage{listings}

% Document
\author{Instructor: Xujie Si, TA: Breandan Considine}
\date{Due: Oct. 4th before class}
\title{COMP 597: Assignment \#1}
\begin{document}
    \maketitle
    \noindent This is the first assignment of COMP 597: Automated Reasoning with ML.


    \section{SAT Encoding}

    The following solutions can be obtained using a solver of your choice. Please show all work. Only solutions obtained using a SAT solver will receive credit.\\

    % https://www.cs.rochester.edu/u/kautz/Courses/244autumn2008/Papers/cook-mitchell-1997.pdf
    \noindent \textbf{Emirpimes} (20 points): An \textit{emirp} is a prime whose digits, when reversed, produce a different prime. An \textit{emirpimes} is a semiprime whose reverse is a different semiprime, e.g., $104851039_{10}$. Confirm its prime factors are emirps in bases 2 and 10. Find another such number. What is the largest emirpimes you can find, whose prime factors are distinct emirps in at least two bases?\\

    \noindent \textbf{Bonus} (10 points): A \textit{cryptarithm} is a cipher, $\varphi: \{A,\ldots, Z\}\rightarrow \{0, \ldots, 9\}$, alongside a meaningful string, whose ciphertext satisfies some equation, e.g.:

    \begin{lstlisting}[basicstyle=\scriptsize\ttfamily]
NINETEEN + THIRTEEN + THREE + TWO + TWO + ONE + ONE + ONE = FORTYTWO
42415114 + 56275114 + 56711 + 538 + 538 + 841 + 841 + 841 = 98750538
    \end{lstlisting}

    \noindent Construct a 20+-character cryptarithm parseable by the following grammar,
    \begin{align*}
E &\rightarrow A \mid \ldots \mid Z \mid EE \mid E O E \mid ( E )\\
O &\rightarrow + \mid \times \mid \div \mid -\footnotemark\\
S &\rightarrow E = E
    \end{align*}
    \footnotetext{Interpreted in the usual way, but additive and multiplicative identity are forbidden.}\noindent where $\texttt{eval}\big(\varphi(E)\big) = \texttt{eval}\big(\varphi(E')\big)$ and $\texttt{charset}(E) \neq \texttt{charset}(E')$. Every plaintext word should be defined in the English 10k dictionary.\footnote{\tiny\url{https://github.com/first20hours/google-10000-english/blob/master/google-10000-english.txt}} In order to receive credit, it must not be possible to find your cryptarithm (or algebraic rewritings thereof) on the internet or in other classmates' assignments.\\

    \pagebreak


    \section{Problem 2: Build or Improve a SAT Solver}

    \noindent \textbf{Programming exercise} (40 points): Please select one of the following two options, write a short report, and submit your source code. Please provide instructions for how reproduce your findings and a few test cases.

    \begin{enumerate}
        \item Write a SAT solver from scratch by implementing an existing algorithm such as DPLL, unit propagation or two-watched literals, describe your implementation and evaluate it on a few toy SAT problems.
        \item Make a substantive improvement to a competitive SAT solver (e.g. Kissat or MiniSat) which measurably increases performance on a standard benchmark, and document your approach and findings.
    \end{enumerate}

    % https://courses.cs.washington.edu/courses/cse599a2/15wi/
    \section {Problem 3: Uninterpreted function equivalence}

    \noindent \textbf{SMT exercise} (40 points): Show all work to receive full credit. Where required, typeset a proof sketch using \LaTeX, then translate the proof into your favorite SMT solver to construct a specific example or counterexample.

    \begin{enumerate}
        \item A polynomial equation whose coefficients and solutions are integers is called \textit{diophantine}. Let $w, x, y, z \in \mathbb{Z}$ and report your solver's largest nontrivial solutions to each of the following diophantine equations:\\ (a) $w = x^3+y^3+z^3$ (b) $w^3 + x^3 = y^3+z^3$ (c) $w^z + x^z = y^z + z$.
        \item Prove that $\mathbb{Z}^{n\times n}$ is associative over $\otimes$, and $\otimes$ is distributive over $\oplus$ for some large $n$. \textbf{Bonus} (5 points): Give an example of a nontrivial finite commutative semiring whose elements are matrices and prove it.
        \item A nonnegative matrix whose rows and columns all sum to the same number is called \textit{bistochastic}. Find distinct examples $M_1, M_2: \mathbb{Z}^{n\times n}$ for some large $n$ such that both are nontrivial bistochastic matrices. \textbf{Bonus} (5 points): Is $M_i M_j$ is bistochastic for all bistochastic $M_i, M_j$?
        \item Consider the polynomial kernel $\Delta: (\mathbf{f}, \mathbf{g})\mapsto (\mathbf{f}\cdot\mathbf{g} + r)^q$. The \textit{kernel trick} states $\forall \mathbf{f}, \mathbf{g}: \mathbb{Z}^d$, $\exists \varphi \mid \langle\varphi(\mathbf{f}), \varphi(\mathbf{g})\rangle = \Delta(\mathbf{f}, \mathbf{g})$. Show the kernel trick holds by finding $\varphi$ for some large $r, d, q: \mathbb{N}$. What can we say about $\mathcal O(\langle\varphi, \varphi'\rangle)$ as $d, q \rightarrow \infty$? Is $\Delta$ a metric? Prove or disprove it.
        \item Prove that 1D discrete convolution, $*: (f, g)[x] \mapsto \sum_{s \in S}f[x-s]g[s]$, over $S=[-j, j]$ for some large value $j \in \mathbb{N}$ is translation equivariant. \textbf{Bonus} (10 points): Prove the 2D case for MNIST, i.e., $[0, 255]^{28\times 28}$.
    \end{enumerate}

    \noindent Please submit your answers as a PDF and supplemental work as a ZIP file.
\end{document}