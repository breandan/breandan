%! Author = breandanconsidine
%! Date = 9/19/21

% Preamble
\documentclass[11pt]{article}

% Packages
\usepackage{amsmath}
\usepackage{hyperref}
\usepackage{amsfonts}

% Document
\author{Instructor: Xujie Si, TA: Breandan Considine}
\date{Due: Oct. 4th before class}
\title{A comprehensive treatise on everything\vspace{-2ex}% to see the effect
}
\title{COMP 597: Assignment \#1}
\begin{document}
    \maketitle
    \noindent This is the first assignment of COMP 597: Automated Reasoning with ML.


    \section{SAT Encoding}

    The following solutions can be obtained using a solver of your choice. Please show all work. Only solutions obtained using a SAT solver will receive credit.\\

    % https://www.cs.rochester.edu/u/kautz/Courses/244autumn2008/Papers/cook-mitchell-1997.pdf
    \noindent \textbf{Palindromic primes} (20 points): A \textit{palprime} is a natural number with exactly two factors which are the same written forwards or backwards. Let $1802201963_{10} = PQR$. What are its factors in base-2? What is the largest integer you can find with at least three factors, all of which are distinct binary palindromes? Please describe your solution and its SAT encoding.\\

    \noindent \textbf{Bonus question} (10 points): Every alphanumeric character in the following ciphertext maps to a single case-insensitive letter in the English alphabet. Every plaintext word can be found in the English 10k dictionary.\footnote{\url{https://github.com/first20hours/google-10000-english/blob/master/google-10000-english.txt}}\\

    \noindent \tiny\texttt{1W9G9 QL U 5U30 I91RM0Y JWM PY3GZULQYT 3SC6AMOP1K 2X 1W0 LKLJZCB R9 I4QAN HYN 24G UIPAP1K 12 NZEM826 QDJ08A0314HA 722AL XS5 4DN95L1UDNPDT 7WUJ 3SC6A9OP1K. PX 7W0 5H3M QB RSD IK S4G J228L, 7WZD BKL1MCL RPVA 9E0Y14UVVK IZ3SC9 MHLP0G 1S 4BM HYN CS5M G98QUIA9. PX DS1, 1W9K RP8V 32DJPD40 JS IZ3SCZ WHGNM5 7S 4B0 UDN V0BL 5M8PUI80 X2G UVA I4J H G0VUJQE98K BCHVV B01 SX 3SCC2Y 7UBFL. TPE0D W2R WHGN 1WQYFPYT QL, QX 1W2B9 QDJM8AM314U8 7228L UG9 JS L433ZZN, 1WMK RPAV WUEZ 7S B4IL1QJ47M 3UA34VU1PSD X25 JW24TW7.}\normalsize\\

    \noindent What is the plaintext and who originally wrote this? Please describe how you solved it, provide the key and SAT encoding in a language of your choice.

    \pagebreak


    \section{Problem 2: Build or Improve a SAT Solver}

    \noindent \textbf{Programming exercise} (40 points): Please select one of the following two options, write a short report, and submit your source code. Please provide instructions for how reproduce your findings and a few test cases.

    \begin{enumerate}
        \item Write a SAT solver from scratch by implementing an existing algorithm such as DPLL, unit propagation or two-watched literals, describe your implementation and evaluate it on a few toy SAT problems.
        \item Make a substantive improvement to a competitive SAT solver (e.g. Kissat or MiniSat) which measurably increases performance on a standard benchmark, and document your approach and findings.
    \end{enumerate}

    % https://courses.cs.washington.edu/courses/cse599a2/15wi/
    \section {Problem 3: Uninterpreted function equivalence}

    \noindent \textbf{SMT exercise} (40 points): Please typeset a proof sketch using \LaTeX, then translate the proof into your favorite SMT solver to construct a specific example or counterexample. Show all work to receive full credit.

    \begin{enumerate}
        \item A polynomial equation whose coefficients and solutions are integers is called \textit{diophantine}. Let $w, x, y, z \in \mathbb{Z}$ and report your solver's largest nontrivial solutions to each of the following diophantine equations: (a) $w = x^3+y^3+z^3$ (b) $w^3 + x^3 = y^3+z^3$ (c) $w^z + x^z = y^z + z$.
        \item Prove that $\mathbb{Z}^{n\times n}$ is associative over $\otimes$ and distributive over $\oplus$ for some large $n$. \textbf{Bonus} (5 points): Give an example of a nontrivial finite commutative semiring whose elements are matrices and prove it.
        \item A nonnegative matrix whose rows and columns all sum to the same number is called \textit{bistochastic}. Find distinct examples $M_1, M_2: \mathbb{Z}^{n\times n}$ for some large $n$ such that both are nontrivial bistochastic matrices. \textbf{Bonus} (5 points): Is $M_i M_j$ is bistochastic for all bistochastic $M_i, M_j$?
        \item Consider the polynomial kernel $\Delta: (\mathbf{f}, \mathbf{g})\mapsto (\mathbf{f}\cdot\mathbf{g} + r)^q$. The \textit{kernel trick} states $\forall \mathbf{f}, \mathbf{g}: \mathbb{Z}^d$, $\exists \varphi \mid \langle\varphi(\mathbf{f}), \varphi(\mathbf{g})\rangle = \Delta(\mathbf{f}, \mathbf{g})$. Show the kernel trick holds by finding $\varphi$ for some large $r, d, q: \mathbb{N}$. What can we say about $\mathcal O(\langle\varphi, \varphi'\rangle)$ as $r, d \rightarrow \infty$? Is $\Delta$ a metric? Prove or disprove it.
        \item Prove that 1D discrete convolution, $*: (f, g)[x] \mapsto \sum_{s \in S}f[x-s]g[s]$, over $S=[-j, j]$ for some large value $j \in \mathbb{N}$ is translation equivariant. \textbf{Bonus} (10 points): Prove the 2D case for MNIST, i.e., $[0, 255]^{28\times 28}$.
    \end{enumerate}

    \noindent Please submit your answers as a PDF and supplemental work as a ZIP file.
\end{document}